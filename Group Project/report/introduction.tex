% !TeX root = report.tex
% !TeX spellcheck = en_GB

\section{Introduction}
Finding the shortest path in a graph is one of the most well studied algorithmic graph problems and solutions for the problem origin back to the late 1950's.\todo[color=green]{maybe ref?} Finding the shortest path has a long list of applications from transport and routing tables in networks to being an essential part of many more advanced graph algorithms. Most problems can be modelled as a shortest path problem on a large enough graph and as such it can be seen as a general problem solver.\todo[color=green]{check liiiige om det her er bullshit - eller find en reference.}\todo{Måske skal vi begrænse os til at sige en subproblem-solver? (og måske henvise til max-flow/min-cut hvor bfs bruges internt?)} In recent years neural networks has gained more and more traction in the artificial intelligence and machine learning fields. This change is both due to the more widespread adoption and advancement of graphical processing units but also due to theoretical breakthroughs.\todo[color=green]{ref?} Neural networks are used for classification, finding heuristics and as the decision mechanism for artificial agents among others.

\newpar One of the most often mentioned problems with neural networks and machine learning in general is that they are inherently single purpose and do not handle different problems and domains very well from what they are trained to do.\todo[color=green]{skal her refereres til den artikel som Sebastian sendte omkring planning problems?}

\newpar With the introduction of neural networks with memory more domains should be feasibly solvable. In \cite{graves2016hybrid} a differentiable neural computer was shown to be able to find the shortest path in the London underground by using memory to store variables and information about the graph.

\newpar In this report we want to examine how well these neural networks with memory can handle the shortest path problem on undirected, unweigthed general graphs.\todo{Genskriv dette når vi har lavet forsøgene, så vi ikke lover mere end vi kan holde.}