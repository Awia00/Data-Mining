% !TeX root = report.tex
% !TeX spellcheck = en_GB

\section{Introduction}
Finding the shortest path in a graph is one of the most well studied algorithmic graph problems. Solutions for the problem origin back to the late 1950's. Finding the shortest path has a long list of applications from transport and routing tables in networks to being an essential subroutine of many more advanced graph algorithms. 
In recent years neural networks has gained more and more traction in the artificial intelligence and machine learning fields. This increased focus on neural networks is both due to theoretical breakthroughs but also the more widespread adoption and advancement of graphical processing units. Neural networks are most often used for classification, finding heuristics and as the decision mechanism for artificial agents.

\newpar One of the most criticized problems of neural networks and machine learning in general is that agents are inherently single purpose and problems different from what is trained on is often not handled very well.

\newpar With the introduction of neural networks with memory more domains should be feasibly solvable. In \cite{graves2016hybrid} a differentiable neural computer was shown to be able to find the shortest path in the London underground by using memory to store variables and information about the graph. In \cite{greve2016evolving} an evolutionary version of the neural Turing machine was introduced. Evolutionary neural networks does not require every part of the problem to be differentiable and allows reward based learning.

\newpar In this report we want to examine how well these evolutionary neural Turing machines can handle the shortest path problem on undirected, unweighted graphs. Both acyclic and recurrent neural networks will be examined.

\newpar The source code of the experimentation can be found at\\ \href{https://github.com/Awia00/Data-Mining/tree/master/Group%20Project/NeatBFS/src/NeatBFS}{www.github.com/Awia00/Data-Mining}.